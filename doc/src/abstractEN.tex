This paper aim to make some research about automation in web penetration testing and to build a lab to test some scripts. The goals are to be able to setup a vulnerable machine with Ansible scripts and then attack it with tools. A detection script will also be used in order to detect and maybe stop such attacks.
In order to do so the cloud of the Vilnius University faculty will be used to have the appropriate machines and setup.
The work will never be really complete and there are some upgrading possible through time, this project is a minimal one in order to have a solid base.
The focus is on web penetration testing since it is one of the most common one but it could be extended to cover more protocols afterwards.

Keywords : penetration testing, automation, cybersecurity, web, vulnerabilities

\newpage

\sectionWithoutNumber{Research and implementation of cyber attacks}

The aim of this work is to develop code and scripts to create a virtual cybersecurity lab using a vulnerable web application and an attacker's computer. To do this, Ansible scripts and Python as a scripting language are used. The main objective is to be able to carry out research on the automation of penetration testing and the use of common tools to exploit common vulnerabilities. A small part of the project will also focus on detecting and blocking such attacks. The motivation is both pedagogical and the research objectives are to improve the knowledge on automated penetration testing and attack detection. The following topics have been analysed in detail: XSS injection, SQL injection and the most common tools that allow attacks. The Ansible syntax and scripts and specifically vulnerable web applications were also covered. Penetration testing in general has been studied to understand each concern and each part of it in order to reproduce it as accurately as possible as manual testing. The lab was properly equipped and tested for SQL, XSS attack and directory listing. Detection was tested in a minimal way using a small script, the main reason being that the web application is so vulnerable that it should be fixed first and only then think about detecting and blocking malicious requests. The results showed that automation was not so simple, and the output of the tools had to be formatted in some way. Setting up a lab using Ansible was not an easy task either, as there are small details that can make things go wrong. Nevertheless, the lab works well, although it is a candidate for many updates, as it covers more attacks and allows for a more detailed detection scenario or the use of special software.