To conclude, this project is not well finished yet and require some improvements to be really exhaustive. However, it offers a base to start from with an easy and fast way to set up a cybersecurity lab in order to do some training for academic purposes. The automation of penetration testing is already a big concern in the cyber industry and most professionals use tools like Nessus to find the most common CVE with a web interface that allows them to know straight away the level of vulnerabilities that they will find on the attack surface. However, the approach of this project is mostly educative, therefore the CLI tools and the description of what is happening is interesting to understand what is happening and why, even though the best way to do that is by going manual. The detection part looks a bit harder because it implies some network knowledge and it seems to be difficult to code by itself, however it definitely could be achieved. There are a lot of open source tools on internet and checking the code and modifying it to meet the need of this project could definitely be interesting even if time consuming. \\

For the personnal part, I would say I really enjoyed this project because it was interesting mostly to play around with ansible but also to discover new tools, however I think I do not have enough knowledge about the tool I used to make them all work properly.